\section{Analysis of other markup languages}

\subsection{Introduction}

In order to decide the syntax of a \acro{HTML} markup language, it is important to examine previous languages, and build on top of their strengths, and try and address their weaknesses. The three main sources used are Jade, a templating built for Node.JS\cite{Jade}, Handlebars.js, a templating language originally built in JavaScript, and has been ported to various other languages, and finally \LaTeX{}, which while not a web markup language, has been in use since 1985\cite{LaTeX}, and thus has a lot more maturity in it's syntax decisions. In addition to research into markup languages, research was done looking into articles written about templating languages by it's users, especially articles that were against the use of templating languages, as it's important to understand templating languages within a larger context, and whether they solve what they set to solve\cite{AgainstTemplating}\cite{LinkedinTemplating}. The main goal of \languageName{}, and it's syntax, is to show a clear, concise, and consistent language providing separation of logic that dictates the form, from content.

\subsection{Jade}

Unlike handlebars, Jade features syntactic sugar for writing html. Jade removes explicit tags, and instead relies on whitespace to determine hierarchy. This allows it to be very terse, lightweight, and simple. A basic example is seen in Figure~\ref{fig:JadeHelloWorld}. However its whitespace dependency can lead to a lot of cases where it's syntax starts to become a hindrance. A simple example of this is shown in Figure~\ref{fig:JadeProblem} where you have paragraph with ``Hello World!'', and within the paragraph the ``World!'' is underlined, and the ``!'' is in bold. This example is arbitrary, but it shows Jade's problems with handling text. Another distinct example of Jade's problem with text is how it handles multi line text, shown in Figure~\ref{fig:JadeMultiLine}. 

Jade assumes the first word in a given line is the name of the element. This resulted in a special character to allow for multi line text, but this can lead to weird indentation to use inline text elements. While \you{} can write normal html elements in the text elements, this solution however leads to inconsistency in the syntax. Even with those two solutions \you{} would still have to make sure that if there is a newline character, that it maintains the same level of indentation. With Jade you have to conform to it's style, rather than have an ideal format suiting to the content. Jade's control flow statements are for the most part the same as JavaScript. Which in the context of being a templating on top of a JavaScript engine, but the file also contained html to solve the previous problem, leads to a file containing three different syntaxes in an attempt to solve the problem of separating logic from content, and in reality only moves the problem into Jade's files, and syntax.

\begin{figure}[ht!]
    \large{\acro{HTML}}\normalsize{}
    \begin{verbatim}
    <!DOCTYPE html>
    <html>
      <body>
        <p>Hello World!</p>
      </body>
    </html>
    \end{verbatim}
    \large{Jade}\normalsize{}
    \begin{verbatim}
    doctype html
    html
        body
            p Hello World!
    \end{verbatim}
    \caption{Hello World in Jade.}
    \label{fig:JadeHelloWorld}
\end{figure}

\begin{figure}[ht!]
    \large{\acro{HTML}}\normalsize{}
    \begin{verbatim}
    <!DOCTYPE html>
    <html>
      <body>
        <p>Hello <u>World<strong>!</strong></u></p>
      </body>
    </html>
    \end{verbatim}
    \large{Jade}\normalsize{}
    \begin{verbatim}
    doctype html
    html
        body
            p Hello
                u World
                    strong !
    \end{verbatim}
    \caption{Text problems in Jade.}
    \label{fig:JadeProblem}
\end{figure}


\begin{figure}[ht!]
    \large{\acro{HTML}}\normalsize{}
    \begin{verbatim}
    <!DOCTYPE html>
    <html>
      <body>
        <p>
          Sing to me of the man, Muse, the man of twists and turns... 
          driven time and again off course, once he had plundered the
          hallowed heights of <strong>Troy</strong>. Many cities of 
          men he saw and learned their minds, many pains he suffered, 
          heartsick on the open sea, fighting to save his life and 
          bring his comrades home.
        </p>
        <p>
          Sing to me of the man, Muse, the man of twists and turns... 
          driven  time and again off course, once he had plundered
          the hallowed heights of <strong>Troy</strong>. Many cities
          of men saw and learned their minds, many pains he suffered, 
          heartsick on the open sea, fighting to save his life and 
          bring his comrades home.
        </p>
      </body>
    </html>
    \end{verbatim}
    \large{Jade}\normalsize{}
    \begin{verbatim}
    doctype html
    html
      body
        p.
          Sing to me of the man, Muse, 
          the man of twists and turns ... 
          driven time and again off course, once he had plundered 
          the hallowed heights of <strong>Troy</strong>. 
          Many cities of men he saw and learned their minds, 
          many pains he suffered, heartsick on the open sea, 
          fighting to save his life and bring his comrades home. 
        p
          | Sing to me of the man, Muse, 
          | the man of twists and turns ... 
          | driven time and again off course, once he had plundered 
          | the hallowed heights of 
          strong Troy 
          | . Many cities of men he saw and learned their minds, 
          | many pains he suffered, heartsick on the open sea, 
          | fighting to save his life and bring his comrades home. 
    \end{verbatim}
    \caption{Multi line problems with Jade.}
    \label{fig:JadeMultiLine}
\end{figure}


\subsection{Handlebars}

Handlebars syntax, is at opposites with Jade, it's syntax is verbose, while designed as a \acro{HTML} templating language, it is in reality language agnostic. As long as \you{} is careful, Handlebars could work on top of any language including \languageName{}. Handlebars doesn't allow for ``explicit'' logic, calling itself a ``logic-less'' language. Where logic is based on the data type of variables passed to the compiler. It also has a feature named ``helper functions'', which allow for custom logic on a variable, and these must be written in an outside JavaScript(or equivalent language that has a port of Handlebars) file, which can provide a good separation of logic from content.

Despite being self proclaimed ``logic-less'', in reality it still uses a lot of logic, just being solely based on the object. Even if \you{} can't write ``if 5 > 10 \{...\}'', \you{} can still write ``\{\{\#if variable}}...\{\{/if\}\}'', making the logic feel very overly verbose.

\begin{figure}[ht!]
    \large{\acro{HTML}}\normalsize{}
    \begin{verbatim}
    <!DOCTYPE html>
    <html>
      <body>
        <p>Hello World!</p>
      </body>
    </html>
    \end{verbatim}
    \large{Handlebars}\normalsize{}
    \begin{verbatim}
    <!DOCTYPE html>
    <html>
      <body>
        <p>{{text}}</p>
      </body>
    </html>
    \end{verbatim}
    \large{JSON}\normalsize{}
    \begin{verbatim}
    {
        "text": "Hello World!"
    }
    \end{verbatim}
    \caption{Hello World in Handlebars.}
    \label{fig:HandlebarsHelloWorld}
\end{figure}

\begin{figure}[ht!]
    \large{\acro{HTML}}\normalsize{}
    \begin{verbatim}
    <!DOCTYPE html>
    <html>
      <body>
        <p>Hello <u>World<strong>!</strong></u></p>
      </body>
    </html>
    \end{verbatim}
    \large{Handlebars}\normalsize{}
    \begin{verbatim}
    <!DOCTYPE html>
    <html>
      <body>
      {{#if hello}}
        <p>{{hello}} <u>{{world}}<strong>{{mark}}</strong></u></p>
      </body>
      {{/if}}
    </html>
    \end{verbatim}
    \large{JSON}\normalsize{}
    \begin{verbatim}
    {
        "hello": "Hello",
        "world": "World",
        "mark": "!"
    }
    \end{verbatim}
    \caption{Verboseness in Handlebars.}
    \label{fig:HandlebarsProblem}
\end{figure}


\subsection{\LaTeX{}}

\LaTeX{}'s base syntax is very lightweight, and text first unlike Jade, making ideal for use within \acro{HTML}, and is the main inspiration for the syntax for \languageName{}. [Unfinished]

\begin{figure}[ht!]
  \begin{verbatim}
  \section{Hello World}
  Sing to me of the man, Muse, 
  the man of twists and turns ... 
  driven time and again off course, once he had plundered 
  the hallowed heights of \textbf{Troy}. 
  Many cities of men he saw and learned their minds, 
  many pains he suffered, heartsick on the open sea, 
  fighting to save his life and bring his comrades home.
  \end{verbatim}
  \caption{Sample \LaTeX{} program.}
\end{figure}

\section{Syntax of \languageName{}}

\subsection{Elements}
All elements are started with a ``/'' character. The  reason for the ``/'' over \LaTeX{}'s ``\\'' is because the character is used for in elements in \acro{HTML}, and provides a familiar character's being used for creating elements. The children of an element are defined within ``{}'' braces. The use of braces to show hierarchy is much more succinct, than \acro{HTML} end tags, and also provides a lot more versatility than being whitespace dependent.

\begin{figure}[ht!]
  \begin{verbatim}
  /html {
      /body {
          /p {
              Hello World!
          }
      }
  }
  \end{verbatim}
  \caption{Hello World in \languageName{}}
\end{figure}

\subsection{Classes, and ids}
Since both the class, and id attribute are the most commonly used attributes in \acro{HTML}, they are given a shortcut syntax in the form of \acro{CSS} selectors. This also provides a very familiar syntax to \you{}, and an easy way to write \acro{HTML} selectors. This feature is mostly inspired by Jade, which has this syntax. 

\begin{figure}[ht!]
  \begin{verbatim}
  /html {
      /body {
          /p.article#helloWorld {
              Hello World!
          }
      }
  }
  \end{verbatim}
  \caption{Classes, and ids in \languageName{}}
\end{figure}


\subsection{Attributes}
This feature is also similar to it's equivalent in Jade, where attributes are defined within ``()'' parameters. It is also possible to enter the ``class'', and ``id'' attribute here if desired. \You{} can enter either single word attributes, like ``required'', or ``contenteditable'', or key value pairings, like ``style'' or ``href''. An element with ``()'' doesn't have to also have braces. This was mainly designed for void elements like ``img'', or ``link'', but can be for any element.

\begin{figure}[ht!]
  \begin{verbatim}
  /html {
      /body {
          /a.article#helloWorld(contenteditable href="index.html") {
              Hello World!
          }
      }
  }
  \end{verbatim}
  \caption{Hello World in \languageName{}}
\end{figure}


\subsection{Variables, and Resources}
Variables are defined with the ``@'' character, as to distinguish identifiers from regular words. Variables can also be attached to elements, these are called resources. This varaible will be the entire contents 